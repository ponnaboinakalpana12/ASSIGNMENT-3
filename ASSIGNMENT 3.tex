
\documentclass[journal,12pt,twocolumn]{IEEEtran}

\usepackage{setspace}
\usepackage{gensymb}

\singlespacing


\usepackage[cmex10]{amsmath}

\usepackage{amsthm}

\usepackage{mathrsfs}
\usepackage{txfonts}
\usepackage{stfloats}
\usepackage{bm}
\usepackage{cite}
\usepackage{cases}
\usepackage{subfig}

\usepackage{longtable}
\usepackage{multirow}

\usepackage{enumitem}
\usepackage{mathtools}
\usepackage{steinmetz}
\usepackage{tikz}
\usepackage{circuitikz}
\usepackage{verbatim}
\usepackage{tfrupee}
\usepackage[breaklinks=true]{hyperref}
\usepackage{graphicx}
\usepackage{tkz-euclide}
\usepackage{float}

\usetikzlibrary{calc,math}
\usepackage{listings}
    \usepackage{color}                                            %%
    \usepackage{array}                                            %%
    \usepackage{longtable}                                        %%
    \usepackage{calc}                                             %%
    \usepackage{multirow}                                         %%
    \usepackage{hhline}                                           %%
    \usepackage{ifthen}                                           %%
    \usepackage{lscape}     
\usepackage{multicol}
\usepackage{chngcntr}

\DeclareMathOperator*{\Res}{Res}

\renewcommand\thesection{\arabic{section}}
\renewcommand\thesubsection{\thesection.\arabic{subsection}}
\renewcommand\thesubsubsection{\thesubsection.\arabic{subsubsection}}

\renewcommand\thesectiondis{\arabic{section}}
\renewcommand\thesubsectiondis{\thesectiondis.\arabic{subsection}}
\renewcommand\thesubsubsectiondis{\thesubsectiondis.\arabic{subsubsection}}


\hyphenation{op-tical net-works semi-conduc-tor}
\def\inputGnumericTable{}                                 %%

\lstset{
%language=C,
frame=single, 
breaklines=true,
columns=fullflexible
}
\begin{document}


\newtheorem{theorem}{Theorem}[section]
\newtheorem{problem}{Problem}
\newtheorem{proposition}{Proposition}[section]
\newtheorem{lemma}{Lemma}[section]
\newtheorem{corollary}[theorem]{Corollary}
\newtheorem{example}{Example}[section]
\newtheorem{definition}[problem]{Definition}

\newcommand{\BEQA}{\begin{eqnarray}}
\newcommand{\EEQA}{\end{eqnarray}}
\newcommand{\define}{\stackrel{\triangle}{=}}
\bibliographystyle{IEEEtran}
\providecommand{\mbf}{\mathbf}
\providecommand{\pr}[1]{\ensuremath{\Pr\left(#1\right)}}
\providecommand{\qfunc}[1]{\ensuremath{Q\left(#1\right)}}
\providecommand{\sbrak}[1]{\ensuremath{{}\left[#1\right]}}
\providecommand{\lsbrak}[1]{\ensuremath{{}\left[#1\right.}}
\providecommand{\rsbrak}[1]{\ensuremath{{}\left.#1\right]}}
\providecommand{\brak}[1]{\ensuremath{\left(#1\right)}}
\providecommand{\lbrak}[1]{\ensuremath{\left(#1\right.}}
\providecommand{\rbrak}[1]{\ensuremath{\left.#1\right)}}
\providecommand{\cbrak}[1]{\ensuremath{\left\{#1\right\}}}
\providecommand{\lcbrak}[1]{\ensuremath{\left\{#1\right.}}
\providecommand{\rcbrak}[1]{\ensuremath{\left.#1\right\}}}
\theoremstyle{remark}
\newtheorem{rem}{Remark}
\newcommand{\sgn}{\mathop{\mathrm{sgn}}}
\providecommand{\abs}[1]{\left\vert#1\right\vert}
\providecommand{\res}[1]{\Res\displaylimits_{#1}} 
\providecommand{\norm}[1]{\left\lVert#1\right\rVert}
%\providecommand{\norm}[1]{\lVert#1\rVert}
\providecommand{\mtx}[1]{\mathbf{#1}}
\providecommand{\mean}[1]{E\left[ #1 \right]}
\providecommand{\fourier}{\overset{\mathcal{F}}{ \rightleftharpoons}}
%\providecommand{\hilbert}{\overset{\mathcal{H}}{ \rightleftharpoons}}
\providecommand{\system}{\overset{\mathcal{H}}{ \longleftrightarrow}}
	%\newcommand{\solution}[2]{\textbf{Solution:}{#1}}
\newcommand{\solution}{\noindent \textbf{Solution: }}
\newcommand{\cosec}{\,\text{cosec}\,}
\providecommand{\dec}[2]{\ensuremath{\overset{#1}{\underset{#2}{\gtrless}}}}
\newcommand{\myvec}[1]{\ensuremath{\begin{pmatrix}#1\end{pmatrix}}}
\newcommand{\mydet}[1]{\ensuremath{\begin{vmatrix}#1\end{vmatrix}}}
\numberwithin{equation}{subsection}
\makeatletter
\@addtoreset{figure}{problem}
\makeatother
\let\StandardTheFigure\thefigure
\let\vec\mathbf
\renewcommand{\thefigure}{\theproblem}
\def\putbox#1#2#3{\makebox[0in][l]{\makebox[#1][l]{}\raisebox{\baselineskip}[0in][0in]{\raisebox{#2}[0in][0in]{#3}}}}
     \def\rightbox#1{\makebox[0in][r]{#1}}
     \def\centbox#1{\makebox[0in]{#1}}
     \def\topbox#1{\raisebox{-\baselineskip}[0in][0in]{#1}}
     \def\midbox#1{\raisebox{-0.5\baselineskip}[0in][0in]{#1}}
\vspace{3cm}
\title{ASSIGNMENT 3}
\author{P.Kalpana}
\maketitle
\newpage
\bigskip
\renewcommand{\thefigure}{\theenumi}
\renewcommand{\thetable}{\theenumi}
Download all python codes from 
\begin{lstlisting}
https://github.com/ponnaboinakalpana12/ASSIGNMENT3
\end{lstlisting}
%
and latex-tikz codes from 
%
\begin{lstlisting}
https://github.com/ponnaboinakalpana12/ASSIGNMENT3
\end{lstlisting}
%
\section{Question No 2.58}
Draw a  pair of tangents to a circle of radius 5 units  which are inclined to each other at an angle of $60\degree$
%
\section{Solution}
Data from the given question :
\numberwithin{table}{section}
\begin{table}[!ht]
\begin{center}
\begin{tabular}{ | m{2cm} | m{1.5cm}| m{2cm} | m{1.5cm} |} 
\hline
& Symbols & Circle \\
\hline
Centre & $\vec{O}$ & \myvec{0\\0} \\ 
\hline
Radius & $r$ & 5\\ 
\hline
\end{tabular}
\end{center}
\end{table}
\\
The angle between the tangents from P given by \theta=60\degree
\begin{lemma}
\label{lemma}
Given a circle of radius r and angle  between the tangents, the intersection of the tangents and points of contact are :
\begin{align*}
P=de_1
\\
where, 
e_1=\myvec{1\\0}, e_2=\myvec{0\\1}\\
\end{align*}
to the circle given by
\begin{align}
x=\frac{r^2}{d}e_1\pm r\sqrt{1-\frac{r^2}{d^2}}e_2\label{eq2.0.1}
\end{align}
If x be a point of contact for tangents 
\begin{align}
PA\perp PB\\
\implies (O-x)^{T}(x-P) &=0\\
or, P^{T}x\implies \norm x^2=r^2\\
\implies e_1^{T}x=\dfrac{r^2}{d}\\
where, d=\frac{r}{\sin\frac{\theta}{2}}\label{eq2.0.6}
\end{align}
Proof:
From \eqref{eq2.0.6}
\begin{align}
\sin{\frac{\theta}{2}} &=\frac{r}{d}
\end{align}

Now substitute the values , 
we get
\begin{align}
d=\frac{r}{\sin\frac{\theta}{2}} \\
\implies d=\frac{5}{\sin30\degree}\\
\implies d=10
\end{align}
and from \eqref{eq2.0.1}
\begin{align}
x=\frac{r^2}{d}\myvec{1\\0}\pm r\sqrt{1-\frac{r^2}{d^2}}\myvec{0\\1}\\
=\frac{25}{10}\myvec{1\\0}\pm 5\sqrt{1-\frac{25}{100}}\myvec{0\\1}\\
x=2.5\myvec{1\\0}\pm 4.33\myvec{0\\1}
\end{align}
\begin{align}
\implies \vec{A}=\myvec{2.5\\4.33}\\
\implies \vec{B}=\myvec{2.5\\-4.33}
\end{align}
from \eqref{lemma}
\begin{align}
P=10\myvec{1\\0}\\
\implies P=\myvec{10\\0}
\end{align}
The coordinates are:
\begin{align}
\therefore \vec{O}=\myvec{0\\0}, \vec{P}=\myvec{10\\0}, \vec{A}=\myvec{2.5\\4.33}, \vec{B}=\myvec{2.5\\-4.33}.
\end{align}
Plot Tangents PA and PB :
\numberwithin{figure}{section}
\begin{figure}[ht]
    \centering
    \includegraphics[width=\columnwidth]{Download.png}
    \caption{Tangent lines to circle of radius 5 units.}
    \label{fig:Tangent lines to circle of radius 5 units.}
\end{figure}    
\end{document}